%! Author = leo
%! Date = 16/10/25

% Preamble
\section{Descrizione del problema e implementazione}
\subsection{Statistiche d'Ordine Dinamico}
Le statistiche d'ordine sono misure che descrivono la distribuzione degli elementi in un insieme ordinato. Nel contesto delle strutture dati dinamiche, ci concentriamo su due operazioni fondamentali:

\begin{itemize}
  \item \textbf{select(k)}: restituisce l'elemento di rango k (il k-esimo elemento più piccolo) nell'insieme.
  \item \textbf{rank(x)}: restituisce il numero di elementi strettamente minori di x nell'insieme.
\end{itemize}

Queste operazioni consentono di estrarre informazioni sulla posizione relativa degli elementi in una collezione ordinata dinamica, dove elementi possono essere aggiunti o rimossi nel tempo.
sono state implementate tre diverse strutture dati per supportare queste operazioni: una lista concatenata ordinata, un albero binario di ricerca standard e un albero binario di ricerca con attributo dimensione.


\subsection{Implementazione}
\subsubsection{Lista Concatenata Ordinata}
La lista concatenata ordinata è implementata con nodi che contengono un valore e un puntatore al nodo successivo. %TODO: decidere come gestire i valori duplicati
\begin{figure}[ht]
  \centering
  \fbox{\rule{0pt}{6cm}\rule{10cm}{0pt}} % altezza 6cm, larghezza 10cm
  \caption{Placeholder: dimensione fissata.}
  \label{fig:fbox}
\end{figure}
\subsubsection{Albero Binario di Ricerca Standard}
L'albero binario di ricerca standard è implementato con nodi che contengono un valore e puntatori ai figli sinistro e destro e al padre.
propietà fondamentale di un albero binario di ricerca è che per ogni nodo, tutti i valori nel sottoalbero sinistro sono minori del valore del nodo, e tutti i valori nel sottoalbero destro sono maggiori.
implementazione tradizionale %TODO: decidere come gestire i valori duplicati
\begin{figure}[ht]
  \centering
  \fbox{\rule{0pt}{6cm}\rule{10cm}{0pt}} % altezza 6cm, larghezza 10cm
  \caption{Placeholder: dimensione fissata.}
  \label{fig:fbox}
\end{figure}
\subsubsection{Albero Binario di Ricerca con Attributo Dimensione}
L'albero binario di ricerca con attributo dimensione estende la struttura del nodo per includere un campo \texttt{size} che tiene traccia del numero di nodi nel sottoalbero radicato in quel nodo.
Questo campo viene aggiornato durante le operazioni di inserimento e cancellazione per mantenere la correttezza delle statistiche d'ordine.
Questa implementazione offre significativi miglioramenti teorici rispetto alle alternative, specialmente quando l'albero non degenera in una lista.
%TODO: decidere come gestire i valori duplicati
\begin{figure}[ht]
  \centering
  \fbox{\rule{0pt}{6cm}\rule{10cm}{0pt}} % altezza 6cm, larghezza 10cm
  \caption{Placeholder: dimensione fissata.}
  \label{fig:fbox}
\end{figure}

\subsection{Complessità Teorica}
\subsubsection{Lista Concatenata Ordinata}
In una lista concatenata ordinata:
\begin{itemize}
  \item \texttt{select(k)} richiede O(k) passaggi, poiché è necessario attraversare la lista dall'inizio fino al k-esimo elemento.
  \item \texttt{rank(x)} richiede O(n) passaggi nel caso peggiore, dove n è la dimensione della lista, poiché potrebbe essere necessario scorrere l'intera lista.
\end{itemize}

\subsubsection{Albero Binario di Ricerca Standard}
In un albero binario di ricerca (BST) non bilanciato:
\begin{itemize}
  \item \texttt{select(k)} richiede O(n) nel caso peggiore, poiché l'implementazione tradizionale richiede una traversata dell'albero che può coinvolgere potenzialmente tutti i nodi.
  \item \texttt{rank(x)} richiede O(n) nel caso peggiore, per le stesse ragioni.
\end{itemize}

Tuttavia, in un albero bilanciato, entrambe le operazioni potrebbero raggiungere O(log n), ma questo non è garantito nell'implementazione standard senza meccanismi di bilanciamento.

\subsubsection{Albero Binario di Ricerca con Attributo Dimensione}
Un BST con campo \texttt{size} che mantiene la dimensione del sottoalbero di ogni nodo:
\begin{itemize}
  \item \texttt{select(k)} richiede O(h) operazioni, dove h è l'altezza dell'albero. Nei casi peggiori di alberi sbilanciati, h può essere O(n), ma in alberi ragionevolmente bilanciati h si avvicina a O(log n).
  \item \texttt{rank(x)} richiede anche O(h), con le stesse considerazioni.
\end{itemize}