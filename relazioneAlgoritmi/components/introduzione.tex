%! Author = leo
%! Date = 16/10/25

% Preamble
\section{Introduzione}
Questa relazione analizza le prestazioni di diverse strutture dati nell'implementazione delle statistiche d'ordine dinamico, in particolare le operazioni di \texttt{select(k)} e \texttt{rank(x)}.

verranno descritte le strutture dati usate, esposte le scelte progettuali fatte e alcuni aspetti implementativi.
Di seguito verranno messe a confronto con opportuni test le prestazioni nelle operazioni
di selezione dell' i-esimo elemento più piccolo e di calcolo del rango di un elemento.
Lo scopo di questa analisi è confrontare tre strutture dati differenti:
\begin{itemize}
  \item Lista concatenata ordinata (\texttt{LinkedList})
  \item Albero binario di ricerca standard senza attributo dimensione (\texttt{BSTree})
  \item Albero binario di ricerca auto bilanciante (AVL) con attributo dimensione (\texttt{SBSTree})
\end{itemize}

Verranno presentati i risultati sperimentali che evidenziano come le diverse implementazioni si comportano al variare delle dimensioni e tipologia di dataset.

\subsection{Esercizio}
% aggiungere snippet dell'esercizio con latex
esercizio assegnato per l'esame di Laboratorio di Algoritmi e Strutture Dati:
\begin{figure}[H]
  \centering
  \begin{tcolorbox}[colframe=black,colback=white,boxsep=8pt,left=6pt,right=6pt,width=\linewidth,sharp corners]
    Vogliamo confrontare varie implementazioni di statistiche d'ordine dinamiche:
    \begin{enumerate}
      \item Con lista ordinata
      \item Con ABR senza attributo \emph{size}
      \item Come visto a lezione (AVL) con attributo \emph{size}
    \end{enumerate}
    Nota: La lista deve essere implementata considerando strutture collegate con puntatori e non la struttura dati lista di Python.
  \end{tcolorbox}
  \caption{Placeholder: dimensione fissata.}
  \label{fig:esercizio}
\end{figure}

