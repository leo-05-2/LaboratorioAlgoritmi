%! Author = leo
%! Date = 16/10/25

% Preamble
\section{test effettuati}

%todo: correggere tabella
i test che sono stati effettuati al momento sono:
- test su dati random per le tre strutture
- test su albero senza size che degenera in lista contro lista
- test su albero senza size bilanciato contro la lista
- test su tutte le strutture con varianza
%todo: test su sbstree bilanciato contro lista
di seguito verranno mostrati i risultati per ogni test effettuato con le relative considerazioni.


\subsection*{Ipotesi Corrette per i Test}

\subsubsection*{Test 1: Dati Casuali}

% Prima immagine (placeholder -> sostituita con rank_performance)
\noindent\makebox[\textwidth][c]{%
  \begin{minipage}{0.7\textwidth}\centering
    \includegraphics[width=\linewidth]{path/to/your/placeholder_image}
    \captionof{figure}{Rank performance (overview)}\label{fig:rank_performance_overview}
  \end{minipage}%
}

Ipotesi Riveduta:

% Due immagini affiancate: LinkedList (random)
\noindent\makebox[\textwidth][c]{%
  \begin{minipage}{0.48\textwidth}\centering
    \includegraphics[width=\linewidth]{../src/rank_performance_LinkedList_random}
    \captionof{figure}{Rank - LinkedList (random)}\label{fig:rank_ll_rand}
  \end{minipage}\hfill
  \begin{minipage}{0.48\textwidth}\centering
    \includegraphics[width=\linewidth]{../src/select_performance_LinkedList_random}
    \captionof{figure}{Select - LinkedList (random)}\label{fig:select_ll_rand}
  \end{minipage}%
}

\texttt{LinkedList}: O(n) (lineare, come previsto)

% BSTree random: due immagini affiancate
\noindent\makebox[\textwidth][c]{%
  \begin{minipage}{0.48\textwidth}\centering
    \includegraphics[width=\linewidth]{../src/rank_performance_BSTree_random}
    \captionof{figure}{Rank - BSTree (random)}\label{fig:rank_bst_rand}
  \end{minipage}\hfill
  \begin{minipage}{0.48\textwidth}\centering
    \includegraphics[width=\linewidth]{../src/select_performance_BSTree_random}
    \captionof{figure}{Select - BSTree (random)}\label{fig:select_bst_rand}
  \end{minipage}%
}

\begin{itemize}
\item Anche se i dati casuali producono mediamente un albero bilanciato con $h\approx\log n$, l'assenza di \texttt{size} costringe gli algoritmi select/rank a visitare/contare elementi.
\item Risultato empirico atteso: le curve di tempo crescono linearmente con $n$, similmente a LinkedList (con costanti diverse).
\end{itemize}

% SBSTree random: due immagini affiancate
\noindent\makebox[\textwidth][c]{%
  \begin{minipage}{0.48\textwidth}\centering
    \includegraphics[width=\linewidth]{../src/rank_performance_SBStree_random}
    \captionof{figure}{Rank - SBSTree (random)}\label{fig:rank_sbst_rand}
  \end{minipage}\hfill
  \begin{minipage}{0.48\textwidth}\centering
    \includegraphics[width=\linewidth]{../src/select_performance_SBStree_random}
    \captionof{figure}{Select - SBSTree (random)}\label{fig:select_sbst_rand}
  \end{minipage}%
}

SBSTree: $O(\log n)$ — grazie al bilanciamento AVL e all'attributo \texttt{size}.

% Grafico comparativo
\noindent\makebox[\textwidth][c]{%
  \begin{minipage}{0.48\textwidth}\centering
    \includegraphics[width=\linewidth]{../src/rank_performance_LinkedList_BSTree_SBStree_random}
    \captionof{figure}{Confronto Rank (random)}\label{fig:rank_cmp_rand}
  \end{minipage}\hfill
  \begin{minipage}{0.48\textwidth}\centering
    \includegraphics[width=\linewidth]{../src/select_performance_LinkedList_BSTree_SBStree_random}
    \captionof{figure}{Confronto Select (random)}\label{fig:select_cmp_rand}
  \end{minipage}%
}

Conseguenza: il test dovrebbe mostrare che LinkedList e BSTree hanno prestazioni comparabili (entrambe $O(n)$), mentre SBSTree è nettamente migliore ($O(\log n)$).

\subsubsection*{Test 2: Dati Ordinati (Albero BSTree Degenere)}

% LinkedList ordinato - affiancate
\noindent\makebox[\textwidth][c]{%
  \begin{minipage}{0.48\textwidth}\centering
    \includegraphics[width=\linewidth]{../src/rank_performance_LinkedList}
    \captionof{figure}{Rank - LinkedList (ordinato)}\label{fig:rank_ll_sorted}
  \end{minipage}\hfill
  \begin{minipage}{0.48\textwidth}\centering
    \includegraphics[width=\linewidth]{../src/select_performance_LinkedList}
    \captionof{figure}{Select - LinkedList (ordinato)}\label{fig:select_ll_sorted}
  \end{minipage}%
}

LinkedList: O(n) (lineare)

% BSTree ordinato
\noindent\makebox[\textwidth][c]{%
  \begin{minipage}{0.48\textwidth}\centering
    \includegraphics[width=\linewidth]{../src/rank_performance_BSTree}
    \captionof{figure}{Rank - BSTree (ordinato)}\label{fig:rank_bst_sorted}
  \end{minipage}\hfill
  \begin{minipage}{0.48\textwidth}\centering
    \includegraphics[width=\linewidth]{../src/select_performance_BSTree}
    \captionof{figure}{Select - BSTree (ordinato)}\label{fig:select_bst_sorted}
  \end{minipage}%
}

BSTree (degenere): O(n)

% SBSTree ordinato
\noindent\makebox[\textwidth][c]{%
  \begin{minipage}{0.48\textwidth}\centering
    \includegraphics[width=\linewidth]{../src/rank_performance_SBStree}
    \captionof{figure}{Rank - SBSTree (ordinato)}\label{fig:rank_sbst_sorted}
  \end{minipage}\hfill
  \begin{minipage}{0.48\textwidth}\centering
    \includegraphics[width=\linewidth]{../src/select_performance_SBStree}
    \captionof{figure}{Select - SBSTree (ordinato)}\label{fig:select_sbst_sorted}
  \end{minipage}%
}

SBSTree: O(\log n) — bilanciamento AVL mantiene $h=O(\log n)$.

% Grafico comparativo finale
\noindent\makebox[\textwidth][c]{%
  \begin{minipage}{0.48\textwidth}\centering
    \includegraphics[width=\linewidth]{../src/rank_performance}
    \captionof{figure}{Confronto Rank (ordinato)}\label{fig:rank_cmp_sorted}
  \end{minipage}\hfill
  \begin{minipage}{0.48\textwidth}\centering
    \includegraphics[width=\linewidth]{../src/select_performance}
    \captionof{figure}{Confronto Select (ordinato)}\label{fig:select_cmp_sorted}
  \end{minipage}%
}

Conseguenza: il test mostra una differenza netta tra BSTree degenerato e SBSTree (che rimane logaritmico).

%todo: considerare come caso test anche numeri che esulano dalla size, e i casi migliori per lista e bstree senza size